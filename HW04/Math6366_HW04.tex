\documentclass{article}

\usepackage{fancyhdr}
\usepackage{extramarks}
\usepackage{amsmath}
\usepackage{amsthm}
\usepackage{amsfonts}
\usepackage{amssymb} % Equations
\usepackage{mathtools}
\usepackage{commath}
\usepackage{tikz}
\usepackage[plain]{algorithm}
\usepackage{algpseudocode}
\usepackage{adjustbox} % Used to constrain images to a maximum size 
\usepackage{xcolor} % Allow colors to be defined
\usepackage{enumerate} % Needed for markdown enumerations to work
\usepackage{geometry} % Used to adjust the document margins
\usepackage{textcomp} % defines textquotesingle
\usepackage[arrow,matrix,curve,cmtip,ps]{xy}
\usepackage{hyperref}

\usetikzlibrary{automata,positioning}

%
% Basic Document Settings
%

\topmargin=-0.45in
\evensidemargin=0in
\oddsidemargin=0in
\textwidth=6.5in
\textheight=9.0in
\headsep=0.25in

\linespread{1.1}

\pagestyle{fancy}
\lhead{\hmwkAuthorName}
\chead{\hmwkClass\ (\hmwkClassInstructor\ \hmwkClassTime): \hmwkTitle}
\rhead{\firstxmark}
\lfoot{\lastxmark}
\cfoot{\thepage}

\renewcommand\headrulewidth{0.4pt}
\renewcommand\footrulewidth{0.4pt}

\setlength\parindent{0pt}


%
% Create Problem Sections
%

\newcommand{\enterProblemHeader}[1]{
    \nobreak\extramarks{}{Problem \arabic{#1} continued on next page\ldots}\nobreak{}
    \nobreak\extramarks{Problem \arabic{#1} (continued)}{Problem \arabic{#1} continued on next page\ldots}\nobreak{}
}

\newcommand{\exitProblemHeader}[1]{
    \nobreak\extramarks{Problem \arabic{#1} (continued)}{Problem \arabic{#1} continued on next page\ldots}\nobreak{}
    \stepcounter{#1}
    \nobreak\extramarks{Problem \arabic{#1}}{}\nobreak{}
}

\setcounter{secnumdepth}{0}
\newcounter{partCounter}
\newcounter{homeworkProblemCounter}
\setcounter{homeworkProblemCounter}{1}
\nobreak\extramarks{Problem \arabic{homeworkProblemCounter}}{}\nobreak{}

%
% Homework Problem Environment
%
% This environment takes an optional argument. When given, it will adjust the
% problem counter. This is useful for when the problems given for your
% assignment aren't sequential. See the last 3 problems of this template for an
% example.
%
\newenvironment{homeworkProblem}[1][-1]{
    \ifnum#1>0
        \setcounter{homeworkProblemCounter}{#1}
    \fi
    \section{Problem \arabic{homeworkProblemCounter}}
    \setcounter{partCounter}{1}
    \enterProblemHeader{homeworkProblemCounter}
}{
    \exitProblemHeader{homeworkProblemCounter}
}


%
% Various Helper Commands
%

% Useful for algorithms
\newcommand{\alg}[1]{\textsc{\bfseries \footnotesize #1}}
% For derivatives
\newcommand{\deriv}[1]{\frac{\mathrm{d}}{\mathrm{d}x} (#1)}
% For partial derivatives
\newcommand{\pderiv}[2]{\frac{\partial}{\partial #1} (#2)}
% Integral dx
\newcommand{\dx}{\mathrm{d}x}
% Alias for the Solution section header
\newcommand{\solution}{\textbf{\large Solution}}


%-------------------------------------------
%       Begin Local Macros
%-------------------------------------------
\newcommand{\Gal}{\mathrm{Gal}}
\newcommand{\Aut}{\mathrm{Aut}}
\newcommand{\Prob}{\mathbf{P}}
\newcommand{\Pow}{\mathcal{P}}
\newcommand{\F}{\mathcal{F}}
\newcommand{\M}{\mathcal{M}}
\newcommand{\A}{\mathcal{A}}
\newcommand{\B}{\mathcal{B}}
\newcommand{\E}{\mathcal{E}}
\newcommand{\n}{\noindent}
\newcommand{\Z}{\mathbb{Z}}
\newcommand{\N}{\mathbb{N}}
\newcommand{\Q}{\mathbb{Q}}
\newcommand{\R}{\mathbb{R}}
\newcommand{\C}{\mathbb{C}}
\newcommand{\T}{\mathbb{T}}
\newcommand{\im}{\operatorname{im}}
\newcommand{\coker}{\operatorname{coker}}
\newcommand{\ind}{\operatorname{ind}}
\newcommand{\rank}{\operatorname{rank}}
\newcommand\mc[1]{\marginpar{\sloppy\protect\footnotesize #1}}
\newcommand{\ra}{\rangle}
\newcommand{\la}{\langle}
%-------------------------------------------
%       end local macros
%-------------------------------------------


%
% Homework Details
%   - Title
%   - Due date
%   - Class
%   - Section/Time
%   - Instructor
%   - Author
%

\newcommand{\hmwkTitle}{Homework 04}
\newcommand{\hmwkDueDate}{Oct 30, 2018}
\newcommand{\hmwkClass}{Math 6366 Optimization}
\newcommand{\hmwkClassTime}{}
\newcommand{\hmwkClassInstructor}{Andreas Mang}
\newcommand{\hmwkAuthorName}{\textbf{Jonathan Schuba}}

%
% Title Page
%

\title{
	\textmd{\textbf{\hmwkClass:\ \hmwkTitle}}\\
	\normalsize\vspace{0.1in}\small{Due\ on\ \hmwkDueDate}\\
}
\author{\hmwkAuthorName}
\date{}

\renewcommand{\part}[1]{\textbf{\large Part \Alph{partCounter}}\stepcounter{partCounter}\\}


\begin{document}

\maketitle

\begin{homeworkProblem}[1]
	Consider the problem
	\[
	\begin{aligned}
		\underset{x}{\text{minimize}} \quad & f_0(x) \\ 
		\text{subject to} \quad & Ax = b 
	\end{aligned} 
	\]
	
	The auxiliary function has the form:
	
	\[
	\phi(x) = f_0(x) + \beta p(x) = f_0(x) + \beta \| Ax-b \|_2^2
	\]
	
	Supposing $\tilde{x}$ is a minimizer of $\phi$, show how to find the dual feasible point from $\tilde{x}$.  Find the corresponding lower bound on the optimal value.  
	

	\textbf{Solution:}
	
	\[
	\begin{aligned}
	\phi(x) &= f_0(x) + \beta (Ax-b)^\top (Ax-b) \\
	\phi(x) &= f_0(x) + \beta [x^\top A^\top Ax - 2 (Ax)^\top b + b^\top b] \\
	\phi ' (x) &= f_0' (x) + 2\beta A^\top(Ax-b)
	\end{aligned}
	\]
	
	if $\tilde{x}$ minimizes $\phi$ then it also minimizes:
	
	\[
	L(x, \nu) = f_0(x) + \nu^\top(Ax-b)
	\]
	with $  \nu^\top = 2\beta (A\tilde{x}-b)$, and then:
	
	\[
	\begin{aligned}
	g(\nu)  &= \inf_x (f_0(x) + \nu^\top(Ax-b)\\
			&= f_0(\tilde{x}) + 2\beta \| A\tilde{x}-b \|_2^2
	\end{aligned}
	\]
	This provides a lower bound on $f_{opt}$, since 
	\[ f_0(x) \ge g(nu) = f_0(\tilde{x}) + 2\beta \| A\tilde{x}-b \|_2^2  \]

	
\end{homeworkProblem}

\pagebreak

\begin{homeworkProblem}[2]
	The weak duality inequality, $g opt \le f opt$ , holds when $g opt = −\infty$ and $f opt = \infty$. Show that it holds in the
	two cases below as well.
	\begin{enumerate}[a]
		\item If $f opt = −\infty$ we must have $g opt = −\infty$.
		\item  If $g opt = \infty$ we must have $f opt = \infty$.
	\end{enumerate}
	
	\textbf{Solution:}
	
	\begin{enumerate}[a]
		\item If $fopt = -\infty$, the primal is unbounded below.  Therefore $L(x,\lambda) = f_0(x) + \sum_i \lambda_i f_i(x)$ is unbounded below, and $gopt = -\infty$.
		\item If $gopt = \infty$, the dual is unbounded above, and the primal is infeasible.  To see this, consider if the primal were feasible, so that $f_i(x) \le 0$ for all $i$. Then for $\lambda \ge 0$.
		\[\begin{aligned}
		g(\lambda) &= inf(f_0(x) + \sum_i \lambda_i f_i(x))\\
					&= f_0(\tilde{x}) + \sum_i \lambda_i f_i(x))
		\end{aligned}		\]
		Which implies the dual is bounded from above.
	\end{enumerate}
	
	
\end{homeworkProblem}

\begin{homeworkProblem}[3]
	Derive a dual problem for the unconstrained problem
	\[
	\begin{aligned}
	\underset{x}{\text{minimize}} \quad & \sum_{i=1}^k \|A_i x + b_i\|_2 + 1/2 \| x-x_{\text{ref}} \|_2^2  \\
	\end{aligned}
	\]
	
	\textbf{Solution:}
	
	Introduce variables $y_i$ and rewrite:
	\[
	\begin{aligned}
	\underset{x}{\text{minimize}} \quad & \sum_{i=1}^k \|y_i\|_2 + 1/2 \| x-x_{\text{ref}} \|_2^2  \\
	\text{subject to} \quad &  A_i x + b_i - y_i = 0 \quad \forall i
	\end{aligned}
	\]
	
	The Lagrangian is:
	\[\begin{aligned}
	L(x,y,\nu) &= \sum_{i=1}^k \|y_i\|_2 + 1/2 \| x-x_{\text{ref}} \|_2^2 + \sum_{i=1}^k \nu_i^\top(A_i x + b_i -y_i)\\
	\text{to find the infimum wrt x, take the derivative}\\
	\nabla_x L(x,y,\nu) &= 0 + x-x_\text{ref} + \sum_{i=1}^k \nu_i^\top A_i \\
	0 &= x-x_\text{ref} + \sum_{i=1}^k \nu_i^\top A_i 
	\end{aligned} \]
	
	So the min of $L$ is $-\infty$ unless $\sum_{i=1}^k \nu_i^\top A_i =0$, whereupon $x = x_\text{ref}$. Then we have:
	
	\[ \begin{aligned}
	g(\nu) &= \inf_y ( \sum_{i=1}^k \|y_i\|_2 + \sum_{i=1}^k \nu_i^\top(A_i x_\text{ref} + b_i - y_i)) \\
	&= \sum_{i=1}^k \nu_i^\top b_i + \sum_{i=1}^k \nu_i^\top A_i x_\text{ref} + \sum_{i=1}^k \inf_y(\|y_i\|_2 - \nu_i^\top y_i) \\
	&= \sum_{i=1}^k [ \nu_i^\top b_i  - \sup_y( \nu_i^\top y_i - \|y_i\|_2 ) ] \quad\quad \text{since}\ \nu_i^\top A_i =0 \\
	&= \sum_{i=1}^k [ \nu_i^\top b_i - \|\nu_i\|_{2*} ]
	\end{aligned} \]
	
	So the dual problem is:
	
		\[
	\begin{aligned}
	\underset{\nu}{\text{maximize}} \quad & g(\nu) \\
	\text{subject to} \quad &  \sum_{i=1}^k \nu_i A_i = 0 \quad
	\end{aligned}
	\]
	
	
\end{homeworkProblem}

\begin{homeworkProblem}[4]
	Consider the problem
	\[
	\begin{aligned}
	\underset{x}{\text{minimize}} \quad & f_0(x) \\ 
	\text{subject to} \quad & f_i(x) \le 0 \quad \forall i
	\end{aligned} 
	\]
	where $f_i$ are differentiable and convex. Suppose $x and \lambda$ satisfy the KKT conditions. Show that this implies that $\nabla f_0 ( x ) ^\top ( x − x^* ) \ge 0$ for all feasible x.
	
	-\linebreak
	\textbf{Solution:}
	
	
	The KKT conditions are:
	\begin{align}
	f_i(x^*) &\le 0 \label{1}\\
	\lambda^*_i &\ge 0 \label{2}\\
	\lambda^*_i  f_i(x^*) &= 0 \label{3} \\
	\nabla f_0(x^*) + \sum \lambda^*_i \nabla f_i(x^*) &= 0 \label{4}
	\end{align} 
	
	If $x$ is feasible, we have:
	\[f_i(x) \le 0\]
	and we have:
	\[\begin{aligned}
		0 \ge f_i(x) &\ge f_i(x^*) + \nabla f_i(x^*)^\top (x-x^*)\\
	0 & \ge \sum \lambda_i^*[ f_i(x^*) + \nabla f_i(x^*)^\top (x-x^*) ] \quad \text{from \eqref{2}}\\
	&= \sum \lambda_i^* f_i(x^*) + \sum \lambda_i^* \nabla f_i(x^*)^\top (x-x^*)  \\
	&= \sum \lambda_i^* \nabla f_i(x^*)^\top (x-x^*) \quad \text{from \eqref{3}} \\
	&= - \nabla f_0(x^*)^\top(x-x^*) \quad \text{from \eqref{4}}
	\end{aligned} \]
	Which establishes the result. 

	
\end{homeworkProblem}

\begin{homeworkProblem}[5]
	Find the dual function of the linear program
	\[
	\begin{aligned}
	\underset{x}{\text{minimize}} \quad & c^\top x \\ 
	\text{subject to} \quad & Gx \preceq h\\
	& Ax = b
	\end{aligned} 
	\]
	Provide the dual problem, and make the implicit equality constraints explicit.
	
	\textbf{Solution:}
	\[\begin{aligned}
	g(\lambda,\nu) &= \inf_x L(x,\lambda, \nu) \\
	&= \inf_x [c^\top x + \lambda^\top(Gx-h) + \nu^\top(Ax-b)] \\
	&= -\lambda^\top h - \nu^\top b + \inf_x(c^\top +\lambda^\top G + \nu^\top A)x \\
	\end{aligned}
	\]
	the latter term is a linear function, so:
	\[
	g(\lambda, \nu) = \begin{cases}
	-\lambda^\top h - \nu^\top b & \text{if}\ \ c^\top +\lambda^\top G + \nu^\top A = 0 \\
	-\infty & \text{otherwise}
	\end{cases}
	\]
	The dual problem is 
	\[
	\begin{aligned}
	\underset{x}{\text{maximise}} \quad & -\lambda^\top h-\nu^\top b\\ 
	\text{subject to} \quad & \lambda \succeq 0 \\
	& c^\top +\lambda^\top G + \nu^\top A = 0
	\end{aligned} 
	\]
	

\end{homeworkProblem}


\begin{homeworkProblem}[7]
	Consider the optimization problem
	\[
	\begin{aligned}
	\underset{x}{\text{minimize}} \quad & tr(Y(x)) \\ 
	\text{subject to} \quad & x \succeq 0\\
	& \mathbf{1}^\top x = 1
	\end{aligned} 
	\]
	where $Y ( x ) : = ( \sum(x_iy_iy_i^\top) )^{-1} $, the vectors $y_i$ are given, and the domain is given by... Derive the dual problem. Simplify the dual problem as much as you can.
	
	\- \linebreak
	\textbf{Solution:}
	
	Rewrite:
	
	\[
	\begin{aligned}
	\underset{x}{\text{minimize}} \quad & tr(X^{-1}) \\ 
	\text{subject to} \quad & -x \preceq 0\\
	& \mathbf{1}^\top x = 1 \\
	& X - \sum(x_iy_iy_i^\top) = 0
	\end{aligned} 
	\]
	
	The Lagrangian is
	
	\[ \begin{aligned}
	L(X,x,\lambda, \nu, N) &= tr(X^{-1}) - \lambda^\top x + \nu(\mathbf{1}^\top x-1) + \langle N, X - \sum(x_iy_iy_i^\top) \rangle \\
	& = tr(X^{-1}) + \rangle N,X\rangle - \sum \lambda_i x_i + \sum \nu x_i - \langle N, \sum(x_iy_iy_i^\top) \rangle -\nu\\
	&= tr(X^{-1}) + tr(NX) + \sum x_i (-\lambda_i + \nu - y_i N y_i^\top) - \nu
	\end{aligned}
	\]
	
	The minimum over $x$ is $-\infty$ unless $-\lambda_i + \nu - y_i N y_i^\top = 0$.
	
	Taking the derivative wrt $X$ yields:
	\[ \begin{aligned}
	\nabla_X L =0 &=  -X^{-2} + N \\
				N &= -X^{-2}\\
				N^{-1/2} &= X
	\end{aligned} \]
	
	The dual function is:
	\[
	g(\lambda,\nu, N ) = \begin{cases}
	2tr(N^{1/2}) - \nu \qquad -\lambda_i + \nu - y_i N y_i^\top = 0 \\
	-\infty \qquad \text{otherwise}
	\end{cases}
	\]
	
	The dual problem is: 
		\[
	\begin{aligned}
	\underset{}{\text{maximise}} \quad & 2tr(N^{1/2}) - \nu\\ 
	\text{subject to} \quad & -\lambda_i + \nu - y_i N y_i^\top = 0 
	\end{aligned} 
	\]
	
	
\end{homeworkProblem}




\begin{homeworkProblem}[8]
	Consider the equality constrained least-squares problem
	\[
	\begin{aligned}
	\underset{x}{\text{minimize}} \quad & \| Ax-b \|_2^2 \\ 
	\text{subject to} \quad & Gx=h\\
	\end{aligned} 
	\]
	Provide the KKT conditions and derive expressions for the primal solution $x^*$ and the dual solution $\nu^*$.
	
	\- \linebreak
	\textbf{Solution:}
	\[\begin{aligned}
	L(x,\nu) &= \| Ax-b \|_2^2 + \nu^\top (Gx-h)\\
	&= x^\top A^\top Ax + (G^\top\nu - 2 A^\top b)^\top x - b^\top b - \nu^\top h \\ 
	\text{taking the derivative} \qquad\qquad
	\nabla_x L = 0 &= 2 A^\top A x + G^\top\nu - 2 A^\top b \\
	x &= 1/2 (A^\top A)^{-1} (2 A^\top b  - G^\top\nu  )	
	\end{aligned} \]
	
	The dual is then:
	\[ g(\nu) = -(1/4)(G^\top\nu - 2A^\top b) ^\top (A^\top A)^{-1} (G^\top \nu - 2A ^\top b) -\nu ^\top h \]
	
	The KKT optimality conditions provide the following equations:
	
	\[
	Gx^* = h \]
	\[
	2 A^\top (Ax^*-b) + G^\top \nu^* = 0
	\]
	
	Solving the equations for $x^*$ and $\nu^*$ yields some very long equations. (Please don't make me type them up.)
	
		
\end{homeworkProblem}



\end{document}
