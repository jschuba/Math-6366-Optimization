\documentclass{article}

\usepackage{fancyhdr}
\usepackage{extramarks}
\usepackage{amsmath}
\usepackage{amsthm}
\usepackage{amsfonts}
\usepackage{amssymb} % Equations
\usepackage{mathtools}
\usepackage{cancel}
\usepackage{commath}
\usepackage{tikz}
\usepackage[plain]{algorithm}
\usepackage{algpseudocode}
\usepackage{adjustbox} % Used to constrain images to a maximum size 
\usepackage{xcolor} % Allow colors to be defined
\usepackage{enumerate} % Needed for markdown enumerations to work
\usepackage{geometry} % Used to adjust the document margins
\usepackage{textcomp} % defines textquotesingle
\usepackage[arrow,matrix,curve,cmtip,ps]{xy}
\usepackage{hyperref}

\usetikzlibrary{automata,positioning}

%
% Basic Document Settings
%

\topmargin=-0.45in
\evensidemargin=0in
\oddsidemargin=0in
\textwidth=6.5in
\textheight=9.0in
\headsep=0.25in

\linespread{1.1}

\pagestyle{fancy}
\lhead{\hmwkAuthorName}
\chead{\hmwkClass\ (\hmwkClassInstructor\ \hmwkClassTime): \hmwkTitle}
\rhead{\firstxmark}
\lfoot{\lastxmark}
\cfoot{\thepage}

\renewcommand\headrulewidth{0.4pt}
\renewcommand\footrulewidth{0.4pt}

\setlength\parindent{0pt}


%
% Create Problem Sections
%

\newcommand{\enterProblemHeader}[1]{
    \nobreak\extramarks{}{Problem \arabic{#1} continued on next page\ldots}\nobreak{}
    \nobreak\extramarks{Problem \arabic{#1} (continued)}{Problem \arabic{#1} continued on next page\ldots}\nobreak{}
}

\newcommand{\exitProblemHeader}[1]{
    \nobreak\extramarks{Problem \arabic{#1} (continued)}{Problem \arabic{#1} continued on next page\ldots}\nobreak{}
    \stepcounter{#1}
    \nobreak\extramarks{Problem \arabic{#1}}{}\nobreak{}
}

\setcounter{secnumdepth}{0}
\newcounter{partCounter}
\newcounter{homeworkProblemCounter}
\setcounter{homeworkProblemCounter}{1}
\nobreak\extramarks{Problem \arabic{homeworkProblemCounter}}{}\nobreak{}

%
% Homework Problem Environment
%
% This environment takes an optional argument. When given, it will adjust the
% problem counter. This is useful for when the problems given for your
% assignment aren't sequential. See the last 3 problems of this template for an
% example.
%
\newenvironment{homeworkProblem}[1][-1]{
    \ifnum#1>0
        \setcounter{homeworkProblemCounter}{#1}
    \fi
    \section{Problem \arabic{homeworkProblemCounter}}
    \setcounter{partCounter}{1}
    \enterProblemHeader{homeworkProblemCounter}
}{
    \exitProblemHeader{homeworkProblemCounter}
}

%
% Homework Details
%   - Title
%   - Due date
%   - Class
%   - Section/Time
%   - Instructor
%   - Author
%

\newcommand{\hmwkTitle}{Homework 02}
\newcommand{\hmwkDueDate}{Sept 25, 2018}
\newcommand{\hmwkClass}{Math 6366 Optimization}
\newcommand{\hmwkClassTime}{}
\newcommand{\hmwkClassInstructor}{Andreas Mang}
\newcommand{\hmwkAuthorName}{\textbf{Jonathan Schuba}}

%
% Title Page
%

\title{
    \textmd{\textbf{\hmwkClass:\ \hmwkTitle}}\\
    \normalsize\vspace{0.1in}\small{Due\ on\ \hmwkDueDate}\\
}
\author{\hmwkAuthorName}
\date{}

\renewcommand{\part}[1]{\textbf{\large Part \Alph{partCounter}}\stepcounter{partCounter}\\}

%
% Various Helper Commands
%

% Useful for algorithms
\newcommand{\alg}[1]{\textsc{\bfseries \footnotesize #1}}
% For derivatives
\newcommand{\deriv}[1]{\frac{\mathrm{d}}{\mathrm{d}x} (#1)}
% For partial derivatives
\newcommand{\pderiv}[2]{\frac{\partial}{\partial #1} (#2)}
% Integral dx
\newcommand{\dx}{\mathrm{d}x}
% Alias for the Solution section header
\newcommand{\solution}{\textbf{\large Solution}}


%-------------------------------------------
%       Begin Local Macros
%-------------------------------------------
\newcommand{\Gal}{\mathrm{Gal}}
\newcommand{\Aut}{\mathrm{Aut}}
\newcommand{\Prob}{\mathbf{P}}
\newcommand{\Pow}{\mathcal{P}}
\newcommand{\F}{\mathcal{F}}
\newcommand{\M}{\mathcal{M}}
\newcommand{\A}{\mathcal{A}}
\newcommand{\B}{\mathcal{B}}
\newcommand{\E}{\mathcal{E}}
\newcommand{\n}{\noindent}
\newcommand{\Z}{\mathbb{Z}}
\newcommand{\N}{\mathbb{N}}
\newcommand{\Q}{\mathbb{Q}}
\newcommand{\R}{\mathbb{R}}
\newcommand{\C}{\mathbb{C}}
\newcommand{\T}{\mathbb{T}}
\newcommand{\im}{\operatorname{im}}
\newcommand{\coker}{\operatorname{coker}}
\newcommand{\ind}{\operatorname{ind}}
\newcommand{\rank}{\operatorname{rank}}
\newcommand\mc[1]{\marginpar{\sloppy\protect\footnotesize #1}}
\newcommand{\ra}{\rangle}
\newcommand{\la}{\langle}
%-------------------------------------------
%       end local macros
%-------------------------------------------




\begin{document}

\maketitle

Answer the following 9 questions and solve the problems on the computational part to get full credit for this homework assignment. Please follow the instructions for homework assignments. I reserve the right to deduct points if you do not follow these rules.

\begin{homeworkProblem}[1]
	Answer the following questions that relate functions f to their epigraph epi f.
	\begin{enumerate}[a]
		\item When is the epigraph of a function a halfspace?
			\subitem When the function is a hyperplane (affine function).
		\item When is the epigraph of a function a convex cone?
			\subitem Functions with the property that $f(ax) = af(x)$.
			It must also include $f(\overrightarrow{0}) = 0$.
		\item When is the epigraph of a function a polyhedron?
			\subitem When the function is a pointwise maximum of affine functions.  Functions like $f(x) = \max\{f_1(x), ... , f_n(x)\}$ have an epigraph which is the intersection of the epigraphs of each $f_i$. If the functions are affine (with halfspace epigraphs), then the intersections of their epigraphs forms a polyhedron. 
	\end{enumerate}
	
\end{homeworkProblem}

\begin{homeworkProblem}[2]
	Here we explore the second-order conditions for convexity on an affine set. Let $F \in \R^{n,m}, \tilde{x} \in \R^n$. The restriction of $f : \R^n \to \R$ to the affine set $\{ Fx+\tilde{x}:x \in \R^m \}$ is defined as the function $\tilde{f} : \R^m \to \R$ with $\tilde{f} (x) = f (Fx + \tilde{x})$, dom $f = \{x: Fx+\tilde{x} \in \mathrm{dom} f \}$. Suppose $f$ is twice differentiable with a convex domain.
	\begin{enumerate}[a]
		\item Show that $\tilde{f}$ is convex if and only if for all $x \in \mathrm{dom} \tilde{f}$ we have that $F^\top \nabla^2 f(Fx + \tilde{x})F \succeq 0$.
		
		\textbf{Solution:}
		
		\begin{align*}
		\tilde{f}(x) &= f(Fx + \tilde{x}) \\
		\nabla   \tilde{f}(x) &= \nabla f(Fx + \tilde{x})F \\
		\nabla^2 \tilde{f}(x) &= F^\top \nabla^2 f(Fx + \tilde{x})F
		\end{align*}
		A function is convex if the Hessian is positive semi-definite.  Therefore, $\tilde{f}$ is convex if $F^\top \nabla^2 f(Fx + \tilde{x})F \succeq 0$, which is the required result. 
			
		\item Suppose that $A \in \R^{p,n}$ is a matrix whose nullspace $\N(A)$ is equal to the range of $F$ and rank $A =	n - $ rank $F$. Show that $\tilde{f}$ is convex if and only if for all $x  \in \mathrm{dom} \tilde{f}$ there exists a $\lambda \in \R$ such that
		$$\nabla^2 f(Fx + \tilde{x}) + \lambda A^\top A \succeq 0$$.
		
		\textbf{Solution:}
		
	\end{enumerate}


	
\end{homeworkProblem}


\begin{homeworkProblem}[3]
	A function $g : \R^n \to \R^n$ is called monotone if for all $x, y \in$ dom $g$,
	
	$$(g(x) - g(y))^\top(x - y) \ge 0$$.
	
	Suppose that $f : \R^n \to \R$ is a differentiable convex function. Show that its gradient $\nabla f$ is monotone.
	
	\textbf{Solution:}
	
	Take $x, y$ in dom $f$. Since $f$ is convex, we know that the following first order conditions hold:
	\begin{align*}
	f(y) &\ge f(x) + \nabla f(x)^\top (y-x) \\
	f(x) &\ge f(y) + \nabla f(y)^\top (x-y) \\
	\shortintertext{Adding these two equations together yields:}
	f(x) + f(y) &\ge f(y) + f(x) + \nabla f(y)^\top (x-y) + \nabla f(x)^\top (y-x) \\
	0 &\ge (\nabla f(x) -\nabla f(y))^\top(y-x)\\
	0 &\le (\nabla f(x) -\nabla f(y))^\top(x-y)
	\end{align*}
	Which is the desired result. 
	
\end{homeworkProblem}

\begin{homeworkProblem}[4]
	We say the function$f :\R^n \times \R^m \to \R$ is convex-concave if $f (x, y)$ is a concave function of $y$, for each fixed $x$, and a convex function of $x$, for each fixed $y$. We also require its domain to have the product form
	dom $f = A \times B$, where $A \in \R^n$ and $B \in \R^m$ are convex.
	\begin{enumerate}[a]
		\item Give a second-order condition for a twice differentiable function $f : \R^n \times \R^m \to \R$ to be convex-concave, in terms of its Hessian $\nabla^2 f (x, y)$.
		
		\textbf{Solution:}
		
		With fixed $\bar{y}$, the function $f(x,\bar{y}) = \bar{f}(x)$ must have positive-semidefinite Hessian, to be convex.  Denote by $\nabla^2_{xx}f(x,y)$ the (pos-semidef) Hessian of size $(n,n)$, which contains partial derivatives of $x$ coordinates only.  
		
		With fixed $\bar{x}$, the function $f(\bar{x},y) = \bar{f}(y)$ must have negative-semidefinite Hessian, to be concave.  Denote by $\nabla^2_{yy}f(x,y)$ the (neg-semidef) Hessian of size $(m,m)$, which contains partial derivatives of $y$ coordinates only.  
		
		Thus, the Hessian of $f(x,y)$ must take the $2\times2$ block form:
		
		 \[
		\nabla^2 f(x,y) =
		\left[ {\begin{array}{cc}
			\nabla^2_{xx}f(x,y) & \nabla^2_{xy}f(x,y) \\
			\nabla^2_{yx}f(x,y) & \nabla^2_{yy}f(x,y) \\
			\end{array} } \right]
		\]
		
		Where $[\nabla^2_{xy}f(x,y)]^\top = \nabla^2_{yx}f(x,y)$ are the rectangular matrices of mixed second partial derivatives. And where $\nabla^2_{xx}f(x,y)$ and $\nabla^2_{yy}f(x,y)$ are pos-semidef and neg-semidef, respectively. 

	\item Suppose that $f :\R^n \times \R^m \to \R$ is convex-concave and differentiable, with $\nabla f (\tilde{x},\tilde{y}) = 0$. Show that the saddle-point property holds: For all $x, y$, we have $f (\tilde{x}, y) \le f (\tilde{x},\tilde{y}) \le f (x,\tilde{y})$. Show that this implies that $f$ satisfies the strong max-min property:
	
	$$	\underset{y\in\R^m}{\sup} \underset{x\in\R^n}{\inf} f(x,y) = \underset{x\in\R^n}{\inf} \underset{y\in\R^m}{\sup} f(x,y)  $$
	
	(and their common value is $f (\tilde{x},\tilde{y})$).
	
	\textbf{Solution:}
	
	Consider $f (x,\tilde{y})$ in terms of a second-order Taylor expansion at $f (\tilde{x},\tilde{y})$.
	\begin{align*}
	f (\tilde{x}, y) \le f (\tilde{x},\tilde{y}) &\le f (x,\tilde{y})\\
		 &\le f (\tilde{x},\tilde{y}) + \nabla f (\tilde{x},\tilde{y})^\top \begin{pmatrix}x-\tilde{x}\\ \tilde{y}-\tilde{y}	\end{pmatrix}  + \frac{1}{2}\begin{pmatrix}x-\tilde{x}\\ \tilde{y}-\tilde{y}	\end{pmatrix}^\top \nabla^2 f(\tilde{x},\tilde{y}) \begin{pmatrix}x-\tilde{x}\\ \tilde{y}-\tilde{y}	\end{pmatrix}\\
		 &\le f (\tilde{x},\tilde{y}) + \cancelto{0}{\nabla f (\tilde{x},\tilde{y})}^\top \begin{pmatrix}x-\tilde{x}\\ 0	\end{pmatrix}  + \frac{1}{2}\begin{pmatrix}x-\tilde{x}\\ 0	\end{pmatrix}^\top \nabla^2 f(\tilde{x},\tilde{y}) \begin{pmatrix}x-\tilde{x}\\ 0 \end{pmatrix}\\
		&\le f (\tilde{x},\tilde{y}) + \frac{1}{2}\begin{pmatrix}x-\tilde{x}\\ 0	\end{pmatrix}^\top \nabla^2 f(\tilde{x},\tilde{y}) \begin{pmatrix}x-\tilde{x}\\ 0 \end{pmatrix}\\
		 &\le f (\tilde{x},\tilde{y}) + \frac{1}{2}\begin{pmatrix}x-\tilde{x}	\end{pmatrix}^\top \nabla^2_{xx} f(\tilde{x},\tilde{y}) \begin{pmatrix}x-\tilde{x}\end{pmatrix}
	\end{align*}
	The second-order term on the right will only involve values in the upper-left block of the Hessian, which we already stated was positive-semidefinite. Thus, the right most term is always greater than or equal to zero, so the right-side inequality holds.  The same argument can be made for the left inequality, which will invoke the negative-semidefinite-ness of $\nabla^2_{yy}f(x,y)$.
	\subitem
	
	Since $f$ is convex in $x$ for fixed $y$, $\underset{x\in\R^n}{\inf} f(x,y)$ occurs where $\nabla f(x,y) = 0$, (which is at $\tilde{x},\tilde{y}$,) so $$\underset{x\in\R^n}{\inf} f(x,y) = f(\tilde{x},y).$$  Similarly, since $f$ is concave in $y$ for fixed $x$, $\underset{y\in\R^m}{\sup}f(\tilde{x},y)$ also occurs at $\tilde{x},\tilde{y}$, so $$\underset{y\in\R^m}{\sup}f(\tilde{x},y) = f (\tilde{x},\tilde{y}).$$
	
	By the same arguments, 
	
	$$ \underset{x\in\R^n}{\inf} \underset{y\in\R^m}{\sup} f(x,y) = \underset{x\in\R^n}{\inf} f(x,\tilde{y}) = f (\tilde{x},\tilde{y}).$$
	
	So, 
	$$	\underset{y\in\R^m}{\sup} \underset{x\in\R^n}{\inf} f(x,y) = f (\tilde{x},\tilde{y}) = \underset{x\in\R^n}{\inf} \underset{y\in\R^m}{\sup} f(x,y)  $$
	\end{enumerate}

	
\end{homeworkProblem}

\begin{homeworkProblem}[5]
	For each of the following functions determine whether it is convex, concave, quasiconvex, or quasiconcave.
	\begin{enumerate}[a]
		\item $f(x) = \exp(x) - 1$ on $\R$.
			\begin{quote}
				$f'(x) = \exp(x)$
				
				$f''(x) = \exp(x) \ge 0$  Therefore, convex.
			\end{quote}
		\item $f (x_1, x_2) = 1/{x_1x_2}$ on $\R^2_{++}$.
			\begin{quote}
				$$ \nabla^2 f(x,y) = \begin{bmatrix}
				2/x^3y&1/x^2y^2  \\ 
				1/x^2y^2& 2/xy^3
				\end{bmatrix}  $$
				
				The function is convex, since this Hessian is positive definite. For any $(a,b) \in \R$,
				\begin{align*}
				& \begin{bmatrix}a & b\end{bmatrix} \begin{bmatrix}
				2/x^3y&1/x^2y^2  \\ 
				1/x^2y^2& 2/xy^3
				\end{bmatrix}  
				\begin{bmatrix}a \\ b\end{bmatrix}  \\
				& = \frac{2a^2}{x^3y} + \frac{2ab}{x^2y^2} + \frac{2b^2}{xy^3}
				\shortintertext{Let $s = \frac{a}{x^{3/2}y^{1/2}}$ and $t = \frac{b}{x^{1/2}y^{3/2}}$.   }
				& = 2 (s^2 + st + t^2)\\
				\shortintertext{Note that $|st| \le s^2 + t^2$ for any $s,t\in\R$, so}
				& = 2 (s^2 + st + t^2)  \ge 0\\
				\end{align*}
				
				
			\end{quote}	
		\item $f (x_1, x_2) = x^2_1/x_2$ on $\R \times \R_{++}$.
			\begin{quote}
				This is quadratic-over-linear, which is convex, from the book. 
			\end{quote}		
	\end{enumerate}
\end{homeworkProblem}

\begin{homeworkProblem}[6]
	Show that $f (X) = tr (X^{-1})$ is convex on dom $f = S^n_{++}$, where $tr:\R^{n,n} \to \R$ is the trace.
	
	\textbf{Solution:}
	
	Let $X \in S^n_{++}$.  Therefore det$(X)>0$, and $X$ admits eigenvalue decomposition:
	
	$$ X = QVQ^{-1} $$ 
	
	Where the diagonal of $V$ holds the eigenvalues of $X$. (ie $V_{ii} = \lambda_i$). Then:
	
	$$ X^{-1} = QV^{-1}Q^{-1} $$
	
	And $V^{-1}_{ii} = 1/\lambda_i$. 
	
	Furthermore, $tr(X) = tr(V)$ and $tr(X^{-1}) = tr(V^{-1})$.  So,
	
	$$ f(X) = tr(X^{-1}) = \sum_{i=1}^{n}\frac{1}{\lambda_i}$$
	$$ f'(X) = \sum_{i=1}^{n}\frac{-1}{\lambda_i^2}$$
	$$ f''(X) = \sum_{i=1}^{n}\frac{2}{\lambda_i^3}$$
	
	Since $X$ is positive definite, all eigenvalues $\lambda_i$ are greater than zero, so the second derivative is always positive.  Therefore $f$ is convex. 
	
\end{homeworkProblem}

\begin{homeworkProblem}[7]
	Show that the following functions are convex. Hint: Use arguments based on composition rules to make your point (i.e., use arguments concerning the characteristics of $h$ and $g$ to establish that $f (x) = h \circ g = h(g(x))$ is convex).
	\begin{enumerate}[a]
		\item $f(x) = -\log(-\log(\sum_{i=1}^{m}\exp(a_i^\top x+b_i))) $ on dom $f = \{ x: \sum_{i=1}^{m}\exp(a_i^\top x+b_i)<1 \}$. Can use fact that log-sum-exp is convex.
		
		\begin{quote}
			Let $g(x) = \log(\sum_{i=1}^{m}\exp(a_i^\top x+b_i))$, which is convex from hint.  Then we have:
			\begin{align*}
			f(x) &= -log(-g(x))  
			\shortintertext{Let $h(x)=-g(x)$, note range $h = [0,\inf)$, from the specification of dom $f$.}
				&= -log(h(x))  
			\end{align*}
			Since log is concave on $\R_{++}$, $-\log$ is convex.  So $f$ is convex.
		\end{quote}
		
		\item $f(x,u,v) = -\log (uv- x^\top x)$ on dom $f = \{ (x,u,v): uv>x^\top x, u,v>0  \}$. Can use fact that $x^\top x/u$ is convex. 
		\begin{quote}
			Since the dom $f$ is all $x,u,v$ where $uv-x^\top x>0$, we can assume that the range of the function $k(x,u,v) = uv-x^\top x$ is all $\R_{++}$ (which is convex).
			\begin{align*}
			f(x,u,v) &= -\log(uv-x^\top x) \\
				&= -\log(k(x,u,v))
			\end{align*}
			Since $-\log$ is convex on $\R_{++}$, $f$ is convex.		
		\end{quote}
		
		\item $f(x,t) = -\log(t^p - \|x\|^p_p)$ where $p>1$ and dom $f = \{ (x,t):t>\|x\|_p \}$.  Can use fact that $\|x\|_p^p/u^{p-1}$ is convex on $\{ (x,t):u>0 \}$
		\begin{quote}
			By the same argument as before, we can assume that since the domain of $f$ is $\{ (x,t):t-\|x\|_p>0 \}$, then $k(x,t)=t-\|x\|_p$ has range over all $\R_{++}$, and is convex.  Since $-\log$ is convex on $\R_{++}$, $f$ is convex. 
		\end{quote}
	\end{enumerate}

	
\end{homeworkProblem}

\begin{homeworkProblem}[8]
	Show that the following statements are valid. Hint: Use arguments based on the fact that the perspective of a convex function is convex to make your point.
	\begin{enumerate}[a]
		\item For $p>1$ the function
		$$ f(x,t ) = \frac{\sum_{i=1}^{n}|x|^p}{t^{p-1}} = \frac{\|x\|_p^p}{t^{p-1}}$$
		is convex on $\{ (x,t) : t>0 \}$.
		
		\begin{quote}
		\textbf{Solution:}
		
		Let $k(x) = \|x\|_p^p$, which is convex since it is a power of a norm, both convex functions.  Then the perspective function applied to $k$ is:
		$$ f(x,t) = t \|x/t\|^p_p = t\frac{\|x\|_p^p}{t^p} =  \frac{\|x\|_p^p}{t^{p-1}} $$.
		Therefore, $f(x,t)$ is the perspective function applied to a convex function, and is convex. 
		\end{quote}
	
		\item The function
		$$ f(x) = \frac{\|Ax+b\|^2_2}{c^\top x+d} $$
		is convex on $\{ x: c^\top x+d > 0 \}$, where $A\in\R^{m,n}, b\in\R^m, c\in\R^n$, and $d\in\R$. 		
		
		\begin{quote}
		\textbf{Solution:}
		
		Take $g(x) = \|Ax+b\|_2$, which is a norm of an affine function, and therefore convex. Take $h(x) = c^\top x+d$, which is affine, and therefore convex (and $>0$ by the problem statement).  We then have:
		
		$$ f(x) = \frac{\|Ax+b\|^2_2}{c^\top x+d} = \frac{g(x)^2}{h(x)}$$
		
		Which is quadratic-over-linear, and is convex.  
		\end{quote}
		
	\end{enumerate}


\end{homeworkProblem}

\begin{homeworkProblem}[9]
	Show that the logistic function $f (x) = \exp(x)/(1 + \exp(x))$ with dom $f=\R$ is log-concave.
	
	\textbf{Solution:}
	
	A function $f$ is log-concave if $f(x) > 0$, and $\ln(f(x))$ is concave.  The first criterion is fulfilled, since we know the logistic function has range $[0,1]$ over domain $\R$, and takes the extreme values only at $\pm\infty$. 
	
	\begin{align*}
	 \ln(f(x)) &= \ln(\exp(x)/(1 + \exp(x))) \\
				&= \ln(\exp(x)) - \ln(1 + \exp(x)) \\
				&= x - \ln(1 + \exp(x))\\
	\deriv{\ln(f(x))}	&= 1-\frac{\exp(x)}{1 + \exp(x)}
	\end{align*}
	This derivative is strictly decreasing, since $\frac{\exp(x)}{1 + \exp(x)}$ is strictly increasing.  A function in $\R$ is concave iff its derivative is monotonically decreasing, so the logistic sigmoid is log-concave.  
	
\end{homeworkProblem}



\end{document}
