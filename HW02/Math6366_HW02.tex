\documentclass{article}

\usepackage{fancyhdr}
\usepackage{extramarks}
\usepackage{amsmath}
\usepackage{amsthm}
\usepackage{amsfonts}
\usepackage{amssymb} % Equations
\usepackage{mathtools}
\usepackage{commath}
\usepackage{tikz}
\usepackage[plain]{algorithm}
\usepackage{algpseudocode}
\usepackage{adjustbox} % Used to constrain images to a maximum size 
\usepackage{xcolor} % Allow colors to be defined
\usepackage{enumerate} % Needed for markdown enumerations to work
\usepackage{geometry} % Used to adjust the document margins
\usepackage{textcomp} % defines textquotesingle
\usepackage[arrow,matrix,curve,cmtip,ps]{xy}
\usepackage{hyperref}

\usetikzlibrary{automata,positioning}

%
% Basic Document Settings
%

\topmargin=-0.45in
\evensidemargin=0in
\oddsidemargin=0in
\textwidth=6.5in
\textheight=9.0in
\headsep=0.25in

\linespread{1.1}

\pagestyle{fancy}
\lhead{\hmwkAuthorName}
\chead{\hmwkClass\ (\hmwkClassInstructor\ \hmwkClassTime): \hmwkTitle}
\rhead{\firstxmark}
\lfoot{\lastxmark}
\cfoot{\thepage}

\renewcommand\headrulewidth{0.4pt}
\renewcommand\footrulewidth{0.4pt}

\setlength\parindent{0pt}


%
% Create Problem Sections
%

\newcommand{\enterProblemHeader}[1]{
    \nobreak\extramarks{}{Problem \arabic{#1} continued on next page\ldots}\nobreak{}
    \nobreak\extramarks{Problem \arabic{#1} (continued)}{Problem \arabic{#1} continued on next page\ldots}\nobreak{}
}

\newcommand{\exitProblemHeader}[1]{
    \nobreak\extramarks{Problem \arabic{#1} (continued)}{Problem \arabic{#1} continued on next page\ldots}\nobreak{}
    \stepcounter{#1}
    \nobreak\extramarks{Problem \arabic{#1}}{}\nobreak{}
}

\setcounter{secnumdepth}{0}
\newcounter{partCounter}
\newcounter{homeworkProblemCounter}
\setcounter{homeworkProblemCounter}{1}
\nobreak\extramarks{Problem \arabic{homeworkProblemCounter}}{}\nobreak{}

%
% Homework Problem Environment
%
% This environment takes an optional argument. When given, it will adjust the
% problem counter. This is useful for when the problems given for your
% assignment aren't sequential. See the last 3 problems of this template for an
% example.
%
\newenvironment{homeworkProblem}[1][-1]{
    \ifnum#1>0
        \setcounter{homeworkProblemCounter}{#1}
    \fi
    \section{Problem \arabic{homeworkProblemCounter}}
    \setcounter{partCounter}{1}
    \enterProblemHeader{homeworkProblemCounter}
}{
    \exitProblemHeader{homeworkProblemCounter}
}

%
% Homework Details
%   - Title
%   - Due date
%   - Class
%   - Section/Time
%   - Instructor
%   - Author
%

\newcommand{\hmwkTitle}{Homework 02}
\newcommand{\hmwkDueDate}{Sept 25, 2018}
\newcommand{\hmwkClass}{Math 6366 Optimization}
\newcommand{\hmwkClassTime}{}
\newcommand{\hmwkClassInstructor}{Andreas Mang}
\newcommand{\hmwkAuthorName}{\textbf{Jonathan Schuba}}

%
% Title Page
%

\title{
    \textmd{\textbf{\hmwkClass:\ \hmwkTitle}}\\
    \normalsize\vspace{0.1in}\small{Due\ on\ \hmwkDueDate}\\
}
\author{\hmwkAuthorName}
\date{}

\renewcommand{\part}[1]{\textbf{\large Part \Alph{partCounter}}\stepcounter{partCounter}\\}

%
% Various Helper Commands
%

% Useful for algorithms
\newcommand{\alg}[1]{\textsc{\bfseries \footnotesize #1}}
% For derivatives
\newcommand{\deriv}[1]{\frac{\mathrm{d}}{\mathrm{d}x} (#1)}
% For partial derivatives
\newcommand{\pderiv}[2]{\frac{\partial}{\partial #1} (#2)}
% Integral dx
\newcommand{\dx}{\mathrm{d}x}
% Alias for the Solution section header
\newcommand{\solution}{\textbf{\large Solution}}


%-------------------------------------------
%       Begin Local Macros
%-------------------------------------------
\newcommand{\Gal}{\mathrm{Gal}}
\newcommand{\Aut}{\mathrm{Aut}}
\newcommand{\Prob}{\mathbf{P}}
\newcommand{\Pow}{\mathcal{P}}
\newcommand{\F}{\mathcal{F}}
\newcommand{\M}{\mathcal{M}}
\newcommand{\A}{\mathcal{A}}
\newcommand{\B}{\mathcal{B}}
\newcommand{\E}{\mathcal{E}}
\newcommand{\n}{\noindent}
\newcommand{\Z}{\mathbb{Z}}
\newcommand{\N}{\mathbb{N}}
\newcommand{\Q}{\mathbb{Q}}
\newcommand{\R}{\mathbb{R}}
\newcommand{\C}{\mathbb{C}}
\newcommand{\T}{\mathbb{T}}
\newcommand{\im}{\operatorname{im}}
\newcommand{\coker}{\operatorname{coker}}
\newcommand{\ind}{\operatorname{ind}}
\newcommand{\rank}{\operatorname{rank}}
\newcommand\mc[1]{\marginpar{\sloppy\protect\footnotesize #1}}
\newcommand{\ra}{\rangle}
\newcommand{\la}{\langle}
%-------------------------------------------
%       end local macros
%-------------------------------------------




\begin{document}

\maketitle

Answer the following 9 questions and solve the problems on the computational part to get full credit for this homework assignment. Please follow the instructions for homework assignments. I reserve the right to deduct points if you do not follow these rules.

\begin{homeworkProblem}[1]
	Answer the following questions that relate functions f to their epigraph epi f.
	\begin{enumerate}[a]
		\item When is the epigraph of a function a halfspace?
		\item When is the epigraph of a function a convex cone?
		\item When is the epigraph of a function a polyhedron?
	\end{enumerate}

	\textbf{Solution:}
	
\end{homeworkProblem}

\begin{homeworkProblem}[2]

	\textbf{Solution:}
	
\end{homeworkProblem}


\begin{homeworkProblem}[3]
	A function $g : \R^n \to \R^n$ is called monotone if for all $x, y \in$ dom $g$,
	
	$$(g(x) - g(y))^\top(x - y) \ge 0$$.
	
	Suppose that $f : \R^n \to \R$ is a differentiable convex function. Show that its gradient $\nabla f$ is monotone.
	
	\textbf{Solution:}
	
	Take $x, y$ in dom $f$. Since $f$ is convex, we know that the following first order conditions hold:
	\begin{align*}
	f(y) &\ge f(x) + \nabla f(x)^\top (y-x) \\
	f(x) &\ge f(y) + \nabla f(y)^\top (x-y) \\
	\shortintertext{Adding these two equations together yields:}
	f(x) + f(y) &\ge f(y) + f(x) + \nabla f(y)^\top (x-y) + \nabla f(x)^\top (y-x) \\
	0 &\ge (\nabla f(x) -\nabla f(y))^\top(y-x)\\
	0 &\le (\nabla f(x) -\nabla f(y))^\top(x-y)
	\end{align*}
	Which is the desired result. 
	
\end{homeworkProblem}

\begin{homeworkProblem}[4]
	
	\textbf{Solution:}
	
\end{homeworkProblem}

\begin{homeworkProblem}[5]
	For each of the following functions determine whether it is convex, concave, quasiconvex, or quasiconcave.
	\begin{enumerate}[a]
		\item $f(x) = \exp(x) - 1$ on $\R$.
		\item $f (x1, x2) = 1/{x_1x_2}$ on $\R^2_{++}$.
		\item $f (x1, x2) = x^2_1/x_2$ on $\R \times \R_{++}$.
	\end{enumerate}

	\textbf{Solution:}
	\begin{enumerate}[a]
		\item $f'(x) = \exp(x)$
		
		$f''(x) = \exp(x) \ge 0$  Therefore, convex.
	\end{enumerate}
	
\end{homeworkProblem}

\begin{homeworkProblem}[6]
	Show that $f (X) = tr (X^{-1})$ is convex on dom $f = S^n_{++}$, where $tr:\R^{n,n} \to \R$ is the trace.
	
	\textbf{Solution:}
	
	Let $X \in S^n_{++}$.  Therefore det$(X)>0$, and $X$ admits eigenvalue decomposition:
	
	$$ X = QVQ^{-1} $$ 
	
	Where the diagonal of $V$ holds the eigenvalues of $X$. (ie $V_{ii} = \lambda_i$). Then:
	
	$$ X^{-1} = QV^{-1}Q^{-1} $$
	
	And $V^{-1}_{ii} = 1/\lambda_i$. 
	
	Furthermore, $tr(X) = tr(V)$ and $tr(X^{-1}) = tr(V^{-1})$.  So,
	
	$$ f(X) = tr(X^{-1}) = \sum_{i=1}^{n}\frac{1}{\lambda_i}$$
	$$ f'(X) = \sum_{i=1}^{n}\frac{-1}{\lambda_i^2}$$
	$$ f''(X) = \sum_{i=1}^{n}\frac{2}{\lambda_i^3}$$
	
	Since $X$ is positive definite, all eigenvalues are greater than zero, so the second derivative is always positive.  Therefore $f$ is convex. 
	
\end{homeworkProblem}

\begin{homeworkProblem}[7]
	
	\textbf{Solution:}
	
\end{homeworkProblem}

\begin{homeworkProblem}[8]
	
	\textbf{Solution:}
	
\end{homeworkProblem}

\begin{homeworkProblem}[9]
	Show that the logistic function $f (x) = \exp(x)/(1 + \exp(x))$ with dom $f=\R$ is log-concave.
	
	\textbf{Solution:}
	
	A function $f$ is log-concave if $f(x) > 0$ and $ln(f(x))$ is concave.  The first criterion is fulfilled, since we know the logistic function has range $[0,1]$ over domain $\R$, and takes the extreme values only at $\pm\infty$. 
	
	\begin{align*}
	 \ln(f(x)) &= \ln(\exp(x)/(1 + \exp(x))) \\
				&= \ln(\exp(x)) - \ln(1 + \exp(x)) \\
				&= x - \ln(1 + \exp(x))\\
	\deriv{\ln(f(x))}	&= 1-\frac{\exp(x)}{1 + \exp(x)}
	\end{align*}
	This derivative is strictly decreasing, since $\frac{\exp(x)}{1 + \exp(x)}$ is strictly increasing.  A function in $\R$ is concave iff its derivative is monotonically decreasing, so the logistic sigmoid is log-concave.  
	
\end{homeworkProblem}



\end{document}
